\documentclass[11pt]{article}
\usepackage[utf8]{inputenc}
\usepackage[margin=2.2in, top=.66in, bottom=.66in]{geometry}
\usepackage{amssymb}
\usepackage{amsmath}

\setlength{\parindent}{0em}
\setcounter{secnumdepth}{0}

\title{}
\date{}

\begin{document}

Rational approximations of irrational numbers offer portable alternatives for computation and theoretical exploration. Today, computer algorithms can approximate numbers but return values of varying length and accuracy.\\

This project measures the viability of three central approximation methods: Dirichlet approximations, continued fractions, and Farey Sequences. The aim is to prove each method and determine which provides the best trade-off between minimizing the length of the approximation and achieving optimal space and time complexities. \\

An \textbf{$\epsilon$-rational approximation} of a real number $\alpha$ is the rational number $\frac{p}{q}$ such that, for some $\epsilon > 0$,
\vspace{-1em}
\begin{align*}
    |\phantom{.}\alpha - \cfrac{p}{q}\phantom{.}|  < \epsilon.
\end{align*}
A \textbf{rational approximation method} is an algorithm that takes input $\alpha$ and $\epsilon$, and returns an $\epsilon$-rational approximation $\frac{p}{q}$. Algorithm code is on github.com/rshahid26/diophantine.\\

A naive approach to approximating a real number is to truncate its decimal expansion. For example, to approximate $\alpha = \pi$ with $\epsilon = 0.005$, we can multiply $\alpha$ with the reciprocal of epsilon and then truncate it.
\begin{align*}
\alpha\left(\cfrac{\frac{1}{\epsilon}}{\frac{1}{\epsilon}}\right) = \pi\left(\cfrac{\frac{1}{0.005}}{\frac{1}{0.005}}\right) = \cfrac{200\pi}{200} = \cfrac{628.31853 \dots}{200} \approx \cfrac{628}{200}
\end{align*}

Therefore, we can define the following rational approximation method with O(1) space and time complexity.\\\\

Note the approximation of $\pi$ uses six digits. Define the number of digits as the \textbf{length}, or $\textit{len}()$, of our approximation. Then, our goal is to find the rational approximation method that minimizes lengths while making the smallest concessions in space and time complexity.\\

For example, when $\alpha=\pi$ and $\epsilon=0.005$, the following $\epsilon$-rational approximations have smaller length.
\begin{align*}
\text{len}\left(\cfrac{22}{7}\right) < \text{len}\left(\cfrac{157}{50}\right) < \text{len}\left(\cfrac{628}{200}\right)
\end{align*}

\newpage
\textbf{Dirichlet Approximation Theorem:} For any real number $\alpha$ and any positive integer $N=\frac{1}{\epsilon}$, there exist integers $p$ and $q$ such that $1 \leq q \leq N$ and $\left| \alpha - \cfrac{p}{q} \right| < \cfrac{1}{qN}$.
\\\\
\textbf{Proof:}

Consider the $N+1$ multiples of $\alpha$: $0, \{ \alpha \}, \{ 2\alpha \}, \dots, \{ N\alpha \}$.
By the Pigeonhole Principle there exists two multiples $m\alpha$ and $n\alpha$ that fall into the same partition.\\


$|-----|-----|---- \cdots ----|-----|$\\
$0\phantom{----..}\frac{1}{N}\phantom{----...}\frac{2}{N}\phantom{-----.}\phantom{---.}\frac{N-1}{N}\phantom{----.}1$ \\
 
Let $m\alpha < n\alpha$. Then, $\{ n\alpha \} - \{ m\alpha \} < \frac{1}{N} = \epsilon$.\\

Now let $p = \lfloor n\alpha \rfloor - \lfloor m\alpha \rfloor$ and $q = n - m$. Since $1 \leq q \leq N$, we can use $m\alpha$ and $n\alpha$ to derive our inequality.
\begin{align*}
    \left| \alpha - \frac{p}{q} \right| &= \left| \frac{q\alpha - p}{q} \right| \\
    &= \left| \frac{n\alpha - m\alpha - (\lfloor n\alpha \rfloor - \lfloor m\alpha \rfloor)}{q} \right| \\
    &= \left| \frac{\{ n\alpha \} - \{ m\alpha \}}{q} \right| < \frac{1}{qN}.
\end{align*}
\textbf{Continued Fractions:} Let $\alpha$ equal the floor of $\alpha$ plus the fractional part of $\alpha$, so $\alpha$ = $\lfloor\alpha\rfloor$ + $\{\alpha\}$. Then, by repeatedly writing $\{\alpha\}$ as $\frac{1}{\frac{1}{\{\alpha\}}}$ and using our equality, we can write any real number as the continued fraction:
\[
\alpha=\lfloor\alpha\rfloor+
\cfrac{1}{\lfloor\frac{1}{\{\alpha\}}\rfloor
+\cfrac{1}{\lfloor\frac{1}{\{\frac{1}{\{\alpha\}}\}}\rfloor
+\cfrac{1}{\lfloor\frac{1}{\{\frac{1}{\{\frac{1}{\{\alpha\}}\}}\}}\rfloor
+\cdots}}}
\]

\newpage
\textbf{Farey Fractions:} A Farey sequence $F_n$ is the set of reduced fractions $p/q$ such that $0\leq p \leq q \leq n$. Then, for consecutive fractions $\frac{a}{b}$ and $\frac{c}{d}$ in $F_n$, their \textbf{mediant} $\frac{a+c}{b+d}$ is between them in $F_n$ if $b+d \leq n$.\\

\textbf{Proof:} We know that $\frac{a}{b} < \frac{c}{d}$, so $\frac{a}{b} - \frac{c}{d} = \frac{ad-bc}{bd} < 0$ and $ad < bc$. To show $\frac{a}{b} < \frac{a+c}{b+d} < \frac{c}{d}$, repeat the process.
\begin{align*}
\frac{a}{b} < \frac{a+c}{b+d} &\iff ab+\underline{ad} < ab+\underline{bc} \\
&\iff ad < bc.
\end{align*}

Similarly, we can show $\frac{a+c}{b+d} < \frac{c}{d}$. This property now lets us binary search a Farey fraction in $F_n$ within $\epsilon$ distance of $\alpha$.\\

Farey sequences of size 1 though 9, color coded. Each subsequent edge is found using the mediant property.\\

By Dirichlet's Approximation Theorem, an $\epsilon$-rational approximation necessarily exists within N=$\frac{1}{\epsilon}$. Below are the time and space complexities of the three algorithms in terms of N.\\

Note the complexity of the continued fraction approximation method depends on how fast the fraction converges to $\alpha$. This is measured by $k$, the number of $\frac{1}{\frac{1}{\{\alpha\}}}$ divisions necessary to get within $\epsilon$. This algorithm has time complexity of $O(k^2)$. \\

In a worst-case scenario, k is proportional to N. However, continued fractions converge very quickly to $\alpha$ so on average $O(k^2)$ is between $O(\log N)$ and $O(N)$.\\

To determine which algorithm minimizes length, I randomly generated 100,000 real numbers $\alpha$ and approximated them to the nearest hundred thousandth, being $\epsilon$. My simulation shows Farey Sequence approximations minimize length in the best time and space complexities.\\

Continued fractions are a close second. In my simulation k $\approx$ 0.0005N giving it an average time complexity of $O(\log N)$.\\

\begin{thebibliography}{9}
\bibitem{rosen2005elementary}
Kenneth H. Rosen,
\textit{Elementary Number Theory},
Pearson/Addison Wesley, 2005.
\end{thebibliography}
Special thanks to my mentor, Ajmain Yamin.
\end{document}

\subsection{}


\end{document}
